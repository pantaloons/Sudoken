Upon running the Sudoken application, you will be presented with the GUI shown in figure TODO. Initially there will be no sudoken puzzle loaded. To load a puzzle, you should click the "Load Puzzle" button, and navigate to the location where your .sudoken puzzle file is stored -- then hit open.

Once you have opened the puzzle, it will be displayed in an unsolved state for inspection. You may then have the sudoken application solve the puzzle by pressing the solve button. You may, at your discretion, choose to load a different puzzle or exit the application at any time using the facilities of your chosen window manager.

While the puzzle is being solved, a sketch of the algorithm progress will be displayed on the puzzle grid for you to observe. The speed at which this algorithm runs can be modified by adjusting the slider at the bottom of the GUI window. The scale of this slider is exponential.

Once a solution to the puzzle has been found, if any exists, the solver will cease to update the board -- showing the final solution. If the board is unchanged from what was loaded, then no solution to the puzzle exists.

You may wish to save the state of the board so that it can be loaded into the application at a later date. This can be done by pressing the save button in the top right corner, and selecting a target file. Typically sudoken puzzles should be saved with the .sudoken file extension.

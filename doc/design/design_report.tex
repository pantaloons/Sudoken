\documentclass[a4paper, 11pt]{article}
%\documentclass[a4paper, 10pt]{amsart}

% multi-line spacing for easier marking
%\usepackage{setspace}

% math symbols
\usepackage{amssymb,amsmath}
\usepackage{textcomp}

% nicer fonts
\usepackage{times}

% for fancy tables with multi-row headers
\usepackage{multirow}
% for pictures
\usepackage{graphicx}

\usepackage{url}
\usepackage{enumitem}
\usepackage{fancyvrb}

% if appending entire pdf docs (e.g. appendix)
%\usepackage{pdfpages}

% narrow margins 
\usepackage[top=2cm, bottom=2cm, left=2cm, right=2cm]{geometry}
% no indention of paragraphs
%\setlength{\parindent}{0in}
% have wider spacing between paragraphs
\setlength{\parskip}{1ex plus 0.5ex minus 0.2ex}
% greater separation between columns
\setlength{\columnsep}{0.7cm}

% handy macro defines
	% code command
\newcommand{\codeCommand}[1]{\texttt{#1}}
	% code 'type' word
\newcommand{\codeType}[1]{\textbf{\texttt{#1}}}

% for code listings, etc.
\usepackage{listings}
\usepackage{color}

\lstset{captionpos=b,tabsize=4,frame=lines,keywordstyle=\color{blue}\bf,commentstyle=\color{OliveGreen},stringstyle=\color{red},numbers=left,numberstyle=\tiny,numbersep=5pt,breaklines=true,showstringspaces=false,basicstyle=\ttfamily\footnotesize,emph={label}}

% titles stuff 
\title{\textbf{Cosc427 Assignment 2} \\ {\textit{Sudoken} - Design Document}}
\author{
	Kevin Doran    \\ 33439377 \and
	Adam Freeth    \\ 68895971 \and
	Timothy Hobbs  \\ 39986601 \and
	Joshua Leung   \\ 46308424 \and
	Michael McGee  \\ 74188139
}

\begin{document}
\maketitle

%\doublespacing % for easier marking

\section{Design Patterns}

\subsection{Model-View-Controller}

The \textit{model-view-controller} pattern is used to completely separate the presentation layer of the program from the domain logic layer, i.e., separating the \textit{view} from the \textit{model}. These layers should be decoupled to allow them to be independent of one another, as the presentation layer should not need to be aware of the logic layer, and vice versa. This allows the graphical user interface to be changed, or a new interface to be added, without the need for modifying any of the logic layer. It similarly allows changes to the logic layer while leaving the presentation layer as it is.

\begin{table}[h!]
\centering
\begin{tabular}{l l}
\textbf{Role} & \textbf{Binding} \\ \hline
Model         & \texttt{Solver} \\
View          & \texttt{SudokenGUI} \\
Controller    & \texttt{Controller} \\
\end{tabular}
\caption{\textit{Model-view-controller} pattern.}
\label{table:mvc}
\end{table}

As shown in Table~\ref{table:mvc}, the \textit{model} is implemented as the \texttt{Solver} interface. The \texttt{Solver} contains a \texttt{Board}, which stores all information relating to the puzzle board and its current state. The method \texttt{Solver.solve()} is able to solve the current puzzle.

The \textit{view} is implemented by the \texttt{SudokenGUI} class. This class is responsible for displaying the application on the screen, including drawing the current puzzle board. It has a reference to \texttt{Controller}, the \textit{controller} that mediates communication between the \textit{model} and \textit{view}. Whenever \texttt{SudokenGUI} wants to perform any action to do with the logic of the program, it calls the relevant method of the \texttt{Controller}, e.g., \texttt{loadPuzzle()} when the user wants to load a new puzzle. \texttt{SudokenGUI} does not know anything about how the logic is performed, other than calling these methods on the \texttt{Controller}.

The \texttt{Controller} class has a reference to both the \texttt{SudokenGUI} and \texttt{Solver}, and controls how the \texttt{Solver} and other domain logic should be handled when the user performs actions. The \texttt{Controller} is unaware of how the GUI works or what the user needs to do to fire events, and is similarly unaware of how the \texttt{Solver} solves the \texttt{Board} or what constitutes a \texttt{Board}.

The use of this pattern means that the GUI does not contain any state, and just delegates all actions to the \textit{controller}. The \textit{controller} interacts with the \textit{model} on behalf of the GUI, and interprets any results, changing the GUI accordingly.

This decoupling allows the user interface and the domain model to be developed and tested independently of one another. It facilitates code reuse and extension, as GUIs and models can be interchanged with alternative implementations. It also makes it easier to develop, as the problem is split up and easier to digest by the designers and programmers.

\subsection{Service Locator}

The \textit{service locator} pattern is used to allow new puzzle extensions to be hotswapped into the application as it is running. For example, a new \texttt{Extension} packaged into a JAR file can be placed into the plugins directory while the application is running, and it will detect that this new \texttt{Extension} has been added and will load these classes, allowing the user to load and solve puzzles of this type.

The \texttt{ExtensionManager} periodically checks the plugin directory for new JAR files. When these new JAR files are found, they are loaded into the application using \texttt{java.util.ServiceLoader}. The \texttt{ExtensionManager} then registers the newly loaded \texttt{Extension}s, placing them in a \texttt{Map} where they are referenced by a string, allowing the program to obtain the appropriate constructors, parsers, and decorators of the extension when needed.

\subsection{Factory Method}

When a puzzle file is loaded into the application, the \texttt{Parser} queries the \texttt{ExtensionManager} to get the \texttt{BoardCreator} of the type of puzzle that is loaded. This \texttt{BoardCreator} is implemented by extensions, such as the \texttt{FutoshikiCreator} for Futoshiki puzzles, and it has a \texttt{create()} method that returns a \texttt{Board}. This is an example of the \textit{factory method} pattern, although the \texttt{Board} class is not extension-specific. Each \textit{ConcreteCreator} instead returns a \texttt{Board} with constraints specific to it, constraints which may or may not be parsed from the puzzle file. The bindings for this pattern in Sudoken are listed in Table~\ref{table:factory}.

\begin{table}[h!]
\centering
\begin{tabular}{l l}
\textbf{Role}   & \textbf{Binding} \\ \hline
Creator         & \texttt{BoardCreator} \\
ConcreteCreator & \texttt{SudokuCreator}, \texttt{FutoshikiCreator}, etc. \\
Product         & \texttt{Board} \\
factoryMethod() & \texttt{create()} \\
\end{tabular}
\caption{\textit{Factory method} pattern.}
\label{table:factory}
\end{table}

The use of this pattern means that the creation of boards and their constraints are delegated to the extensions themselves, and the \texttt{Parser} does not need to know anything about the extension or its constraints, other than the name of the extension (which is specified in the puzzle file). The pattern also allows different extensions to create \texttt{Board}s in their own way, with the functionality of \texttt{create()} being defined in implementations of \texttt{BoardCreator}. Extensions are also able to define and implement their own \texttt{Constraint}s.

\subsection{Strategy}

As a \texttt{Board} is defined by a grid of values and a collection of \texttt{Constraint}s, it is possible to make different implementations of \texttt{Solver}s that will work for any puzzle type. The only requirement is that the entire grid can be filled with values while none of the \texttt{Constraint}s are violated., which can be checked using \texttt{Constraint.isViolated()}. This allows for different strategies to be implemented to solve puzzles, i.e., the \textit{strategy} pattern.

\begin{table}[h!]
\centering
\begin{tabular}{l l}
\textbf{Role}        & \textbf{Binding} \\ \hline
Strategy             & \texttt{Solver} \\
ConcreteStrategyA    & \texttt{BacktrackingSolver} \\
ConcreteStrategyB    & \texttt{SmartBacktrackingSolver} \\
AlgorithmInterface() & \texttt{solve()} \\
\end{tabular}
\caption{\textit{Factory method} pattern.}
\label{table:strategy}
\end{table}

\subsection{Facade}

\subsection{Observer}

\end{document}
